%任务书格式设置
\linespread{1.3}    %页面行距取1.3倍
\songti   %正文使用宋体小4号字
%任务书内容
\noindent
\zihao{-4}%
一、学生姓名: \hspace{20ex}学号:  \\
二、题目:  \\
三、题目来源:自拟  \\
四、结业方式:论文  \\
五、主要内容:\\
\zihao{5}%
\newlength{\mylen}\settowidth{\mylen}{在IPv6网络下,对数据包的排队模型的进行研究并在实际网络中评价模型的}
\indent
\begin{minipage}[t]{\mylen}
  \qquad 在IPv6网络
\end{minipage}\\
\zihao{-4}\vspace{1pt}\\%
六、主要(技术)要求:\\
\zihao{5}%
\indent(1) \\
\indent(2) \\
\indent(3) \\
\indent(4) \\
\zihao{-4}%
七、日程安排:\\
\zihao{5}\indent%
\begin{tabular}{rl}
  % after \\: \hline or \cline{col1-col2} \cline{col3-col4} ...
  1- 2周& 查阅并学习文献,写出文献综述;\\
  3- 4周& 完成并参加选题报告;\\
  5- 8周& 完成……的研究,完成英文文献翻译,填写中期检查表;\\
  9-14周& ……写出毕业论文;\\
  15-16周& 论文送审并参加答辩. \\
\end{tabular}
\vspace{5pt} \\
\zihao{-4}%
八、主要参考文献和书目:\\
\zihao{5}\indent%
\begin{tabular}{ll}
$[1]$& 顾军, 夏士雄, 张瑾. IPv6 环境下的端到端 QoS 模型 [J]. 计算机工程与设\\
     & 计,2007, 28(9):pp 2037-2041.\\
\end{tabular}
\vspace{5ex}
\linespread{2.5}
\begin{flushright}
 \begin{tabular}{rr}
   指导教师签字:&\hspace{30ex}年\hspace{5ex}月\hspace{5ex}日\\
   学生签字:&\hspace{30ex}年\hspace{5ex}月\hspace{5ex}日\\
   系(所)负责人章:&\hspace{30ex}年\hspace{5ex}月\hspace{5ex}日\\
 \end{tabular}
\end{flushright}


